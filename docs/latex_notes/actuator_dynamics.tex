\documentclass[conference]{IEEEtran}
\IEEEoverridecommandlockouts
% The preceding line is only needed to identify funding in the first footnote. If that is unneeded, please comment it out.
\usepackage{pdfpages}
\usepackage{cite}
\usepackage{amsmath,amssymb,amsfonts}
\usepackage{algorithmic}
\usepackage{graphicx}
\usepackage{textcomp}
\usepackage{xcolor}
\usepackage{amsmath}
\usepackage{cleveref}
\def\BibTeX{{\rm B\kern-.05em{\sc i\kern-.025em b}\kern-.08em
    T\kern-.1667em\lower.7ex\hbox{E}\kern-.125emX}}
\begin{document}

\title{TITLE}

\author{\IEEEauthorblockN{Abhay Gupta}
\IEEEauthorblockA{\textit{Department of Mechanical Engineering} \\
\textit{University of Washington}\\
Seattle, WA \\}
}
\maketitle

\begin{abstract}

\end{abstract}

\section{Introduction and Overview}
    Series elastic actuator dynamics models are common for controller design; however, they are rare for trajectory design due to their computational complexity. Lumped models have served well for early development of actuator controllers, but more descriptive models are now necessary. In spite of the increasingly more complex designs and more sophisticated controllers, the same basic lump model has continued to be used almost exclusively.

The general trend when modeling SEAs is to make these assumptions: actuator is treated as the output with the other side treated as ground, the motor and transmission intertias are combined into a single lumped inertia and the actuator dynamics are derived with the motor torque as the input and the force on the assumed output side as the output. They oversimplify the actuator dynamics while overemphasizing the spring dynamics. 



\section{Simulation Model}
    \input{sections/models}

\section{Algorithm Implementation and Development}

\section{Computational Results}

\section{Summary and Conclusions}

\section{Appendix A: Matlab Functions}

There were a few standard MATLAB functions used for implementation of the algorithm. These functions are explained below. 

\section{Appendix B: Matlab Code}

\section{Appendix C: Quick Latex Functions}

\begin{equation}
    A = \alpha I =
    \begin{bmatrix}
        \alpha & 0 \\
        0 & \alpha
    \end{bmatrix}
\end{equation}

\begin{equation}
    ||x_{k+1} - Ax_k||
\end{equation}

\begin{equation}
    \mathbf{X'}=\left[ \begin{array}{cccc}
    {| } & {| } & { } & {|} \\ 
    {\mathbf{x}_{2}} & {\mathbf{x}_{3}} & {\cdots} & {\mathbf{x}_{m}} \\
    {| } & {| } & { } & {|} \\ 
    \end{array}\right]
\end{equation}

\begin{equation}
    \mathrm{A}=\mathrm{X}^{\prime} \mathrm{X}^{\dagger}
\end{equation}

\noindent
where $\dagger$ is the Moore-Penrose pseudo-inverse. This solution minimizes the error:

\begin{equation}
    \left\|\mathbf{X}^{\prime}-\mathbf{A X}\right\|_{F}
\end{equation}

\noindent
where $||\cdot||_F$ is the Frobenius norm, given by

\begin{equation}
    \|\mathbf{X}\|_{F}=\sqrt{\sum_{j=1}^{n} \sum_{k=1}^{m} X_{j k}^{2}}
\end{equation}

\begin{figure}[ht!]
% \centerline{\includegraphics[width=1\columnwidth]{sing_val.png}}
\caption{The singular value spectrum of the original video. }
\label{sing_val}
\end{figure}



\end{document}
